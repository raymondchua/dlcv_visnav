\documentclass[25pt, a0paper, landscape]{tikzposter}

\tikzposterlatexaffectionproofoff
\usepackage[utf8]{inputenc}
\usepackage{authblk}
\makeatletter
\renewcommand\maketitle{\AB@maketitle} % revert \maketitle to its old definition
\renewcommand\AB@affilsepx{\quad\protect\Affilfont} % put affiliations into one line
\makeatother
\renewcommand\Affilfont{\Large} % set font for affiliations
\usepackage{amsmath, amsfonts, amssymb}
\usepackage{tikz}
\usepackage{pgfplots}
% align columns of tikzposter; needs two compilations
\usepackage[colalign]{column_aligned}



% tikzposter meta settings
\usetheme{Default}
\usetitlestyle{Default}
\useblockstyle{Default}

%%%%%%%%%%% redefine title matter to include one logo on each side of the title; adjust with \LogoSep
\makeatletter
\newcommand\insertlogoi[2][]{\def\@insertlogoi{\includegraphics[#1]{#2}}}
\newcommand\insertlogoii[2][]{\def\@insertlogoii{\includegraphics[#1]{#2}}}
\newlength\LogoSep
\setlength\LogoSep{-70pt}

\renewcommand\maketitle[1][]{  % #1 keys
    \normalsize
    \setkeys{title}{#1}
    % Title dummy to get title height
    \node[inner sep=\TP@titleinnersep, line width=\TP@titlelinewidth, anchor=north, minimum width=\TP@visibletextwidth-2\TP@titleinnersep]
    (TP@title) at ($(0, 0.5\textheight-\TP@titletotopverticalspace)$) {\parbox{\TP@titlewidth-2\TP@titleinnersep}{\TP@maketitle}};
    \draw let \p1 = ($(TP@title.north)-(TP@title.south)$) in node {
        \setlength{\TP@titleheight}{\y1}
        \setlength{\titleheight}{\y1}
        \global\TP@titleheight=\TP@titleheight
        \global\titleheight=\titleheight
    };

    % Compute title position
    \setlength{\titleposleft}{-0.5\titlewidth}
    \setlength{\titleposright}{\titleposleft+\titlewidth}
    \setlength{\titlepostop}{0.5\textheight-\TP@titletotopverticalspace}
    \setlength{\titleposbottom}{\titlepostop-\titleheight}

    % Title style (background)
    \TP@titlestyle

    % Title node
    \node[inner sep=\TP@titleinnersep, line width=\TP@titlelinewidth, anchor=north, minimum width=\TP@visibletextwidth-2\TP@titleinnersep]
    at (0,0.5\textheight-\TP@titletotopverticalspace)
    (title)
    {\parbox{\TP@titlewidth-2\TP@titleinnersep}{\TP@maketitle}};

    \node[inner sep=0pt,anchor=west] 
    at ([xshift=-\LogoSep]title.west)
    {\@insertlogoi};

    \node[inner sep=0pt,anchor=east] 
    at ([xshift=\LogoSep]title.east)
    {\@insertlogoii};

    % Settings for blocks
    \normalsize
    \setlength{\TP@blocktop}{\titleposbottom-\TP@titletoblockverticalspace}
}
\makeatother
%%%%%%%%%%%%%%%%%%%%%%%%%%%%%%%%%%%%%


% color handling
\definecolor{TumBlue}{cmyk}{1,0.43,0,0}
\colorlet{blocktitlebgcolor}{TumBlue}
\colorlet{backgroundcolor}{white}

% title matter
\title{End to End Learning for Visual Navigation}

\author[1]{Ute Schiehlen}
\author[1]{Natalie Reppekus}
\author[1]{Raymond Chua}
\author[1]{Areeb Kamran}

\affil[1]{Technical University of Munich}

\insertlogoi[width=15cm]{tum_logo}
\insertlogoii[width=15cm]{tum_logo}

%adaptions
\definecolorpalette{tumPalette}{
	\definecolor{colorOne}{named}{TumBlue}
	\definecolor{colorTwo}{named}{TumBlue}
	\definecolor{colorThree}{named}{TumBlue}
}
\usecolorpalette{tumPalette}

\usetitlestyle{Empty}
\useblockstyle{Minimal}
\useinnerblockstyle{Minimal}
\colorlet{blocktitlefgcolor}{TumBlue}
\colorlet{innerblocktitlebgcolor}{TumBlue}
\colorlet{notebgcolor}{TumBlue}
\colorlet{notefgcolor}{white}



% main document
\begin{document}

\maketitle

\begin{columns}
    \column{0.5}
    \block{Abstract}{
    An important component in autonomous vehicles is visual navigation. We used an  ordered sequence of frames recorded at the front of the car to obtain the steering angle using an end to end approach. Based on these images, we also computed the optical flow over adjacent frames to get additional motion information as input to our network and added an LSTM cell after the convolutional layers. 
    
    End to end learning using {\em fully convolutional networks} (FCNs) has proven to be successful in many computer vision tasks such as Recognition\cite{DBLP:journals/corr/SermanetEZMFL13}, Object Detection\cite{DBLP:journals/corr/SermanetEZMFL13}, Localization\cite{DBLP:journals/corr/SermanetEZMFL13, conf/cvpr/OquabBLS15} and Semantic Segmentation \cite{DBLP:journals/pami/ShelhamerLD17}. One reason why end to end learning is so powerful is that it allows the network to learn the internal representation of the data using the provided training information.
    
    Our approach is based on the network proposed in Bojarski et al.\cite{DBLP:journals/corr/BojarskiTDFFGJM16} consisting of five convolutional and three fully connected layers using the human steering angle as training signal. As this model did not fully exploit the temporal information of image sequences, we extended this architecture by training an additional convolutional network with optical flow calculated from the original images as input and optimizing the combined loss. As this approach yielded good results, we trained a second network consisting of a  {\em Long-Short term memory} (LSTM) cell instead of the fully connected layers.
   }
   
    \block{Data Preprocessing}{
    	\innerblock{Images}{
		\coloredbox[width=200pt]{\center RGB}
		\coloredbox[width=200pt]{\center Random Horizontal Flip}
		\coloredbox[width=200pt]{\center Normalize}
		\coloredbox[width=200pt]{\center Convert to YUV}
		\coloredbox[width=200pt]{\center Zero-Mean}
	}
	\innerblock{Labels}{
	}	
    }
    

    \column{0.5}
    \block{Second column first block}{Content in your block.}
    \block{Bibliography}{
    	{\small
		\bibliographystyle{abbrv}
		\bibliography{bib}
	}
    }
\end{columns}

\end{document}
